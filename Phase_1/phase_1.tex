\documentclass[]{article}
\usepackage{color}
\usepackage{graphicx}
\usepackage{fancyhdr}
\usepackage{geometry}
\usepackage{verbatim}
\usepackage[pagebackref=true,colorlinks,linkcolor=blue,citecolor=green]{hyperref}
\usepackage{ptext}
\usepackage{atbegshi}
\usepackage{xepersian}
\settextfont{Zar}
\settextfont{XB Niloofar}
\setlatintextfont{Junicode}

\usepackage[width=0.00cm, height=0.00cm, left=2cm, right=1.5cm, top=2.50cm, bottom=2.5cm]{}

\title{
	\begin{center}
		\includegraphics[width=8cm,height=10cm]{AUT.png}
	\end{center}	
	
	\LARGE
	درس: اصول طراحی نرم افزار \\
	\Large
	استاد: دکتر احسان علیرضایی \\
	\Large
	فاز صفر پروژه (و آپدیت آن)

}
\author{
	گروه 4: \\
	مهدی مهدوی - 9833067 \\
	علیرضا پرورش - 9912013 \\
	آروین اسدی - 9913701 \\
	امیرمحمد کمانی - 9913704
}
\fancypagestyle{logo}{
	\fancyhf{}
	\fancyhfoffset[R]{3.4cm}\fancyfoot[CE,CO]{~\\[.5cm]\thepage}
	\fancyhfoffset[L]{3.4cm}
	\fancyhead[RE,RO]{~\\[-2cm]\includegraphics[height=2.3cm]{MCS}}
	\renewcommand{\headrulewidth}{0pt}
	
}
\pagestyle{logo}

\textheight=21.5cm

\begin{document}
	\maketitle
	\thispagestyle{empty}
	\newpage
	\tableofcontents
	\newpage

	\fontsize{12pt}{14pt}
	\LARGE
%	\selectfont


	\section{\huge{مقدمه}}
	\subsection{\LARGE{معرفی سیستم}}
	سامانه رزومه ساز یک برنامه کاربردی است که به کاربران این امکان را می‌دهد که رزومه‌های حرفه‌ای و جذاب خود را ایجاد و ویرایش کنند. این سیستم باید قابلیت ایجاد بخش‌های مختلف رزومه، وارد کردن اطلاعات شخصی، تجربه کاری، تحصیلی و مهارت‌ها را فراهم کند. همچنین برنامه باید قابلیت انتخاب قالب‌های زیبا و قابل تنظیم را داشته باشد.
	
	
	\subsection{\LARGE{صورت مسئله}}
	
	بسیاری از افراد در جستجوی کار یا فرصت‌های شغلی به رزومه‌های حرفه‌ای نیاز دارند. اما بسیاری از آن‌ها دارای دانش کافی در زمینه طراحی رزومه نیستند و یا نمی‌توانند وقت و انرژی بسیاری را صرف تهیه و ویرایش رزومه کنند. بنابراین، هدف این پروژه طراحی یک نرم‌افزار راحت و کارآمد برای ساخت و ویرایش رزومه‌های حرفه‌ای است که به کاربران امکان می‌دهد به سرعت و با کیفیت بالا رزومه‌های خود را تهیه کنند.
	
	\subsection{\LARGE{هدف}}
	
	هدف اصلی این پروژه ایجاد یک نرم‌ افزار رزومه ساز است که کاربران بتوانند با استفاده از آن به راحتی رزومه‌های خود را طراحی و سفارشی کنند. این نرم‌افزار باید قابلیت‌هایی مانند انتخاب قالب‌های متنوع، وارد کردن اطلاعات شخصی، تجربه کاری، تحصیلی و مهارت‌ها، تنظیم ظاهر رزومه و تبدیل آن به فرمت‌های مختلف را داشته باشد.
	
	\subsubsection{\Large{\lr{GSM}}}
	یکی از اهداف اصلی و مهم در طراحی اپلیکیشن رزومه ساز مربوط به کیفیت رزومه‌های ساخته شده برای کاربران است. 
	کاربران روزمه‌های ساخته شده را جهت اهداف مختلفی همچون معرفی خود در یک جمع عمومی، استخدام در سازمان و ارگان ها، پذیرش در یک تیم و... مورد استفاده قرار می‌دهد.
	بنابراین هدفی که در این باره به دنبال آن تحقق هستیم، جذب شدن هر چه بیشتر مخاطبان مدنظر و درنتیجه پذیرش رزومۀ کاربر در مقاصد مد نظر اوست.
	
	و در جهت دستیابی به این هدف سیگنال زیر را تعریف می‌کنیم:
	
	رزومه‌هایی که ما تولید می‌کنیم برای کاربران، توسط کافرمای مدنظرشان پذیرفته شود.
	
	این سیگنال قابل اندازه گیری با متریک زیر می‌باشد:
	
	اگر حداقل هفتاد درصد از رزومه‌های تولید شده در مقاصد هدف کاربران، مورد پذیرش قرار بگیرند‌، هدف اولیه محقق شده است.
	یا اگر حداقل هشتاد درصد کاربران از رزومه ی تولید شده ی خود رضایت داشتند (امتیاز بالای 4 از 5 دادند.)، هدف ما محقق شده است.
	
	یکی دیگر از اهداف اصلی پروژۀ ساخت رزومه، مربوط به میزان تطبیق قالب و نوع رزومه با اطلاعات و هدف کاربر است. به طور مثال رزومۀ یک معلم جهت استخدام در مدرسه، قالب، ویژگی و ساختاری متفاوت با رزومۀ یک پزشک جهت استخدام در بیمارستان خواهد داشت. 
	لذا هدف دوم مربوط به تطبیق قالب رزومه با اهداف و اطلاعات کاربران می‌توان در نظر گرفت.
	
	
	و برای دستیابی به این هدف سیگنال زیر مناسب می‌باشد:
	
	 با دسته بندی‌هایی که برای قالب‌های رزومه در نظر می‌گیریم با توجه به انواع مشاغل و کارفرما قالب درست انتخاب ‌شود.
	 
	این سیگنال قابل اندازه گیری می‌باشد و می‌توان به صورت زیر آن را در نظر گرفت:
	
 اگر رزومه‌های تولید شده تا 95 درصد با هدف کاربر سنخیت داشته باشد هدف ما محقق شده است.
	
	
	
	
	\section{\huge{نیازهای عملیاتی}}
	
	
	به طور کلی، نیازهای عملیاتی برای یک نرم‌افزار رزومه ساز عبارتند از:
	
	\subsection{\LARGE{رابط کاربری ساده و کارآمد}}
	نرم‌افزار باید دارای یک رابط کاربری کاربر پسند باشد که به کاربران امکان می‌دهد به راحتی رزومه خود را ویرایش و بخش های مختلف سفارشی سازی  کنند. رابط کاربری باید ساده و قابل فهم باشد و کاربران را در انتخاب قالب، وارد کردن اطلاعات و تنظیمات مختلف یاری کند.
	
	\subsection{\LARGE{قالب‌های زیبا و حرفه‌ای}}
	نرم‌افزار باید انواع قالب‌های زیبا و حرفه‌ای برای رزومه‌ها ارائه کند. این قالب‌ها باید به کاربران امکان دهند رزومه‌هایی با ظاهر جذاب و متناسب با نیازهای شغلی و تحصیلی خود ایجاد کنند. همچنین باید امکان شخصی سازی قالب‌ها به منظور تنظیمات شخصی و تغییرات مورد نیاز را فراهم کند.
	
	\subsection{\LARGE{وارد کردن تجربه کاری و تحصیلی}}
	نرم‌افزار باید به کاربران امکان دهد تجربه کاری و تحصیلی خود را به صورت جامع و سازمان‌یافته وارد کنند. این شامل نام شرکت‌ها، عنوان شغلی، تاریخ ها و توصیف کوتاهی از وظایف و دستاوردهای کسب شده و همچنین مهارت های لازم و کسب شده باشد(این بخش در ادامه به صورت کامل توضیح داده شده است)
	
	
	\subsection{\LARGE{وارد کردن مهارت‌ها}}
	نرم‌افزار باید به کاربران امکان دهد مهارت‌های خود را به صورت جامع و دقیق وارد کنند. این شامل مهارت‌های فنی، زبان‌های برنامه‌نویسی، مهارت‌های نرم، مهارت‌های تجاری و سایر مهارت‌های مرتبط با شغل است.
	
	\subsection{\LARGE{نبدیل فرمت فایل}}
	نرم‌افزار باید امکان  تبدیل رزومه به فرمت‌های مختلف را فراهم کند. برخی از فرمت‌های معمول شامل \lr{PDF} و \lr{DOC} است. این امکان به کاربران اجازه می‌دهد رزومه خود را به سادگی به اشتراک بگذارند یا درخواست‌های شغلی را ارسال کنند.
	
	\section{\huge{ابزار ها و فناوری}}
	استفاده از ابزارها و فناوری  استفاده از ابزارها و فناوری‌های زیر نیز ممکن است برای اجرای نیازهای عملیاتی مورد نیاز باشد:
	
	\subsection{\LARGE{پایگاه داده}}
	برای ذخیره و مدیریت اطلاعات کاربران و رزومه‌های آن‌ها، استفاده از یک سیستم مدیریت پایگاه داده مناسب (مانند \lr{MySQL} یا \lr{PostgreSQL}) لازم است.
	
	\subsection{\LARGE{فناوری وب}}
	
	برنامه رزومه ساز باید به صورت یک برنامه وب قابل دسترسی باشد تا کاربران بتوانند از هر مکان و با هر دستگاهی (مانند رایانه شخصی، تبلت و تلفن همراه هوشمند) به آن دسترسی پیدا کنند. از تکنولوژی‌های مانند 
	\lr{HTML}، \lr{CSS}، \lr{JavaScript}، و...  
	جهت پیاده‌سازی بخش‌های سمت کاربر \lr{(Front-end)}، و از فریمورک‌ها و زبان‌های برنامه‌نویسی مانند 
	\lr{Python}، \lr{Ruby}، \lr{PHP} یا \lr{Node.js} برای پیاده‌سازی بخش‌های سمت سرور
	 \lr{(Back-end)} می‌توان استفاده کرد.
	
	
	\subsection{\LARGE{طراحی رابط کاربری}}
	برای ایجاد رابط کاربری ساده و کارآمد، می‌توان از ابزارها و فریمورک‌های طراحی رابط کاربری مانند 
	\lr{Bootstrap}، \lr{Material-UI}، \lr{Tailwind CSS} و یا \lr{React} استفاده کرد.
	
	
	\subsection{\LARGE{تبدیل فرمت فایل}}
	
	
	همچنین برای  تبدیل رزومه به فرمت‌های مختلف مانند \lr{PDF} و \lr{DOC}، می‌توان از کتابخانه‌ها و ابزارهای سروری مانند 
	\lr{Apache PDFBox}، \lr{iText}، و یا استاندارد برنامه‌ نویسی
	 \lr{Office Open XML OOXML} 
	 برای ایجاد فایل‌های \lr{DOCX} استفاده کرد.
	
	\subsection{\LARGE{امنیت}}
	باید توجه ویژه‌ای به امنیت اطلاعات کاربران و رزومه‌های آن‌ها داشته باشید. می‌توان از تکنیک‌ها و استانداردهای امنیتی مانند رمزنگاری، احراز هویت کاربران، و مدیریت دسترسی استفاده کرد.
	
	\subsection{\LARGE{پشتیبانی و نگهداری}}
	برای اطمینان از عملکرد صحیح و پایدار نرم‌افزار رزومه ساز، لازم است که منابع سخت افزاری و نرم‌افزاری مناسب برای پشتیبانی و نگهداری آن در نظر گرفته شود. این شامل سرورهای قدرتمند، پشتیبانی از پایگاه داده‌ها، مانیتورینگ و رفع خطاهای سیستم و به‌روزرسانی‌های منظم نرم‌افزار است.
	
	
	\section{\huge{سخت افزار}}
	
	\subsection{\LARGE{سرور یا سیستم میزبان}}
	
	برای نگهداری و اجرای نرم‌افزار رزومه ساز، شما نیاز به یک سرور یا سیستم میزبان دارید. سرور می‌تواند یک سرور فیزیکی باشد که در دیتاسنتر یا محل مشخصی قرار دارد، یا می‌توانید از سرویس‌های ابری مانند 
	\lr{(Amazon Web Services (AWS)}، \lr{Google Cloud Platform (GCP)}  یا \lr{Microsoft Azure}
	 استفاده کنید. انتخاب سرور یا سیستم میزبان بستگی به مقیاس پروژه و ترافیک تخمینی کاربران شما دارد.
	
	
	\subsection{\LARGE{پردازنده \lr{(CPU)}}}
	پردازش رزومه‌ها و عملیات مختلف در نرم‌افزار نیازمند قدرت پردازش است. پردازنده‌های با عملکرد بالا و تعداد هسته‌های بیشتر می‌توانند بهبود کارایی و سرعت اجرای نرم‌افزار را فراهم کنند.
	
	\subsection{\LARGE{حافظه \lr{(RAM)}}}
	برای بارگذاری و نگهداری اطلاعات در حافظه و اجرای همزمان فرآیندها، حافظه \lr{RAM} مهم است. حداقل حجم \lr{RAM }مورد نیاز برای اجرای نرم‌افزار بستگی به حجم و تعداد کاربران همزمان و مقیاس پروژه شما دارد.
	
	\subsection{\LARGE{فضای ذخیره‌سازی (هارد دیسک)}}
	
	برای ذخیره سازی اطلاعات کاربران، رزومه‌ها و سایر موارد، فضای ذخیره‌سازی لازم است. می‌توان از هارد دیسک‌های با ظرفیت بالا و سرعت خواندن/نوشتن برای این منظور استفاده کرد. همچنین، می‌توان از سرویس‌های ابری برای فضای ذخیره‌سازی مانند \lr{Amazon S3} یا \lr{Google Cloud Storage} استفاده کرد.
	
	\subsection{\LARGE{شبکه}}
	برای دسترسی کاربران به نرم‌افزار رزومه ساز، اتصال به اینترنت و شبکه لازم است. باید از پهنای باند کافی برای پشتیبانی از ترافیک کاربران در همان زمان اطمینان حاصل شود.
	
	\subsection{\LARGE{مانیتورینگ و پشتیبانی}}
	برای نگهداری و پشتیبانی از سرور و نرم‌افزار رزومه ساز، ممکن است نیاز به ابزارها و سرویس‌های مانیتورینگ مانند نرم‌افزارهای مدیریت سرور، رصد و آنالیز عملکرد سیستم و پشتیبانی فنی داشته باشید.
	
	\section{\huge{جمع‌آوری اطلاعات و شناسایی و لیست کردن ذینفعان}}
	
	
	\subsection{\LARGE{تعیین اهداف پروژه}}
	
	ابتدا باید هدف و اهداف پروژه را مشخص کرد. این شامل تعریف محصول نهایی، ویژگی‌های اصلی، و نیازهای کاربران است.
	
	
	\subsection{\LARGE{شناسایی ذینفعان}}
	بررسی شود که کدام افراد یا گروه‌ها در این پروژه تأثیرگذار هستند و به عنوان ذینفعان شمارش می‌شوند. ذینفعان می‌توانند شامل موارد زیر باشند:
	
	\subsubsection{\Large{کاربران نهایی}}
	
	افرادی که از نرم‌افزار رزومه ساز استفاده خواهند کرد؛ مانند افراد دنبال کننده شغل و افرادی که در حال جستجوی کار هستند.
	
	
	\subsubsection{\Large{تیم توسعه}}
	اعضای تیم توسعه نرم‌افزار رزومه ساز، شامل برنامه‌نویسان، طراحان و مدیران پروژه.
	
	\subsubsection{\Large{مدیران و مالکان}}
	افرادی که مسئولیت نهایی پروژه را دارند و نیازمندی‌های خود را دارند.
	
	\subsubsection{\Large{مشاوران و متخصصان}}
	افرادی که در زمینه‌های مرتبط با نرم‌افزار رزومه ساز تخصص دارند و می‌توانند راهنمایی کنند.
	
	\subsubsection{\Large{سایر ذینفعان}}
	ممکن است گروه‌ها یا سازمان‌های دیگری نیز از پروژه تأثیر بگیرند، مانند بخش منابع انسانی یک شرکت.
	
	
	\subsection{\LARGE{تحلیل و سازماندهی اطلاعات}}
	بعد از جمع‌آوری اطلاعات، آن‌ها را تحلیل کنید و در یک ساختار منظم سازماندهی کنید. می‌توانید از مدل‌های مختلفی مانند نمودارهای شبکه، نمودارهای سازمانی یا در ادامه می‌توانید از روش‌های زیر برای جمع‌آوری اطلاعات و شناسایی و لیست‌کردن ذینفعان استفاده کنید:
	
	
	\subsubsection{\Large{مصاحبه}}
	با ذینفعان مختلف مصاحبه کنید تا نیازها و توقعاتشان را درباره نرم‌افزار رزومه ساز بفهمید. می‌توانید مصاحبه‌های شخصی، تلفنی یا آنلاین را انجام دهید. در نهایت، سوالات استانداردی برای هر گروه ذینفعان تهیه کنید تا از همه جوانب پروژه آگاه شوید.
	
	\subsubsection{\Large{نظرسنجی}}
	از طریق فرم‌های آنلاین یا ارسال پرسشنامه‌ها، نظرسنجی را به منظور جمع‌آوری نظرات و بازخوردهای ذینفعان انجام دهید. این روش می‌تواند مفید باشد در مواقعی که نیاز به نظرات گسترده‌تری از ذینفعان دارید.
	
	
	\subsubsection{\Large{تحلیل مستندات}}
	مستندات موجود مانند طرح کسب و کار، گزارش‌ها، دستورالعمل‌ها و مستندات مرتبط دیگر را بررسی کنید. این مستندات می‌توانند اطلاعات مفیدی در مورد نیازها و توقعات ذینفعان فراهم کنند.
	
	
	\subsubsection{\Large{گروه‌های مشارکتی}}
	تشکیل گروه‌های مشارکتی با ذینفعان مختلف می‌تواند به شما کمک کند تا اطلاعات بیشتری درباره نیازها و توقعاتشان کسب کنید. در این گروه‌ها، ذینفعان می‌توانند با یکدیگر و با شما به اشتراک بگذارند و در فرآیند طراحی نرم‌افزار رزومه ساز شرکت کنند.
	
	\subsubsection{\Large{تحلیل فرایند کسب و کار}}
	با تحلیل فرایند کسب و کار، می‌توانید روند استفاده از نرم‌افزار رزومه ساز را در محیط کسب و کار درک کنید و نیازها و توقعات ذینفعان را در این فرایند شناسایی کنید.
	
	
	
	با توجه به اطلاعاتی که از ذینفعان جمع‌آوری کرده‌اید، می‌توانید یک لیست از ذینفعان را تهیه کنید و بر اساس اهمیت و تأثیر آن‌ها بر پروژه، اولویت‌بندی کنیم
	
	
	\section{\huge{نیازمندی ها}}
	\subsection{\LARGE{سناریو: ثبت‌نام کاربران}}
	
	
	
	\textbf{انجام‌دهنده:}
	
	کاربر جدید
	
	\textbf{هدف:}
	
	کاربر جدید با ثبت‌نام در سایت می‌تواند برای ساخت روزمه خود اقدام کند.
	
	\textbf{نیازمندی‌های \lr{Functioanl}:} 
	
	۱- برای دریافت نام‌کاربری، ایمیل و رمز ورود کاربر نیاز به یک فرم ثبت‌نام داریم.
	
	۲- صحت فرمت آدرس ایمیل و تکراری نبودن آدرس ایمیل باید بررسی شود.
	
	۳- پسورد باید دو بار وارد شود و هر دو بار یکسان باشد.
	
	۴- پسورد باید امینت داشته باشد و شامل حروف بزرگ و کوچک و اعداد و کاراکترهای دلخواه باشد.
	
	۵- پس از ثبت‌نام، باید یک ایمیل حاوی لینک تایید برای کاربر فرستاده شود.
	
	۶- کاربران باید پس از کلیلک روی لینک تاییدی که به ایمیلشان فرستاده شده، ثبت‌‌نامشان کامل شود.
	
	۷- پس از کامل شدن ثبت نام، کاربر باید به صورت خودکار درون سایت \lr{Login} شود.
	
	۸- امکان ثبت‌نام با استفاده از حساب \lr{Google} نیز وجود دارد. کاربر روی دکمه «ثبت‌نام با \lr{Goolge}» کلیک کند.
	
	\textbf{نیازمندی‌های \lr{Non-Functional}:}
	
	۱- فرم ثبت‌نام نباید گیج‌کننده باشد.
	
	۲- در صورت موفقیت یا شکست خوردن عملیات ثبت‌نام، رنگ نوشته‌ها باید مطابق انتظار باشد. (مثلا سبز برای موفقیت و قرمز برای خطا)
	
	۳- روال ثبت‌نام باید امن و فارغ از هرگونه نشت اطلاعات حساس کاربری باشد.
	
	۴- روال ثبت‌نام باید در برابر جملاتی مانند \lr{SQL Injection} مقاوم باشد.
	
	۵- سامانه باید ایمیل حاوی لینک تایید ثبت‌نام را فورا برای کاربر ارسال کند.
	
	\textbf{ارتباط سناریو با نیازمندی‌ها:}
	
	در  این سناریو روال ساخت حساب جدید برای کاربر مشخص می‌شود. سعی می‌شود کابر بتواند از طرق مختلف حساب کاربری جدیدی را بسازد و در این مسیر گیج نشود و همه کارها به صورت امن انجام شوند.
	
	
	\subsection{\LARGE{سناریو: ورود کاربران به سایت}}
	
	\textbf{انجام‌دهنده:}
	
	کاربر ثبت‌نام کرده در سایت
	
	\textbf{هدف:}
	
	کاربر می‌تواند وارد حساب کاربری خود شود. رزومه خود را بسازد یا آن را ویرایش کند.
	
	
	
	
	
	\textbf{نیازمندی‌های \lr{Functioanl}:} 
	
	۱-یک فرم ورود برای دریافت نام کاربری و رمز عبور مورد نیاز است.
	
	۲- با کلیک روی گزینه «ورود با رمز یکبار مصرف» در صورتی که کاربر شماره موبایل خود درون اطلاعات کاربری ثبت‌ کرده باشد،‌ یک رمز یکبار مصرف برای وی ارسال می‌شود و می‌تواند با استفاده از آن وارد سایت شود.
	
	۳- پس از ورود موفق به سایت کاربر باید به صفحه اصلی سایت هدایت شود.
	
	۴- در صورت وارد کردن اطلاعات غلط،‌باید پیغام خطای مناسب به کاربر نمایش داده شود.
	
	۵- امکان ورود به سایت با استفاده از حساب \lr{Google} وجود دارد.
	
	\textbf{نیازمندی‌های \lr{Non-Functional}:}
	
	۱- فرم ورود نباید گیج‌کننده باشد.
	
	۲- در صورت موفقیت یا شکست خوردن عملیات ثبت‌نام رنگ نوشته‌ها باید مطابق ان
	تظار باشد. (مثلا سبز برای موفقیت و قرمز برای خطا)
	
	۳- روال ثبت‌نام باید امن و فارغ از هرگونه نشت اطلاعات حساس کاربری باشد.
	
	۴- روال ثبت‌نام باید در برابر حملاتی مانند \lr{SQL Injection} مقاوم باشد.
	
	
	\textbf{ارتباط سناریو با نیازمندی‌ها:}
	
	در  این سناریو روال کاربر به حساب خود توضیح داده شده است. در این بخش حفظ اطلاعات کاربری و احراز هویت دقیق از اهمیت بالایی برخوردار است تا مبادا دسترسی غیرمجاز به حساب فرد صورت پذیرد
	
	
	\subsection{\LARGE{سناریو: خروج از حساب کاربری}}
	
	\textbf{انجام‌دهنده:}
	
	کاربر وارد شده در سایت
	
	\textbf{هدف:}
	
	کاربر می‌تواند وارد حساب کاربری خود شود. رزومه خود را بسازد یا آن را ویرایش کند.
	
	\textbf{نیازمندی‌های \lr{Functioanl}:} 
	
	۱- در قسمت اطلاعات کاربر که در \lr{Header} نمایش داده می‌شود باید دکمه \lr{Logout} وجود داشته باشد.
	
	۲- با کلیک بر روی دکمه خروج، کاربر از حساب کاربری خود خارج شده و به صفحه اصلی سایت هدایت می‌شود.
	
	\textbf{نیازمندی‌های \lr{Non-Functional}:}
	
	۱- فرآیند خروج نباید پیچیده و طولانی باشد.
	
	۲- برای پیشگیری از نشت اطلاعات باید اطمینان حاصل شود که خروج کاربر از سایت توسط سرور تایید می‌شود.
	
	
	\textbf{ارتباط سناریو با نیازمندی‌ها:}
	
	
	در این سناریو کاربر به صورت امن از حساب خود خارج می‌شود.
	
	\subsection{\LARGE{سناریو: ویرایش اطلاعات کاربری}}
	
	\textbf{انجام‌دهنده:}
	
	کاربر وارد شده در سایت
	
	
	\textbf{هدف:}
	
	کاربر ‌می‌تواند اطلاعات خصوصی خود مانند رمز عبور،‌ آدرس ایمیل و شماره تلفن خود را تغییر دهد.
	
	\textbf{نیازمندی‌های \lr{Functioanl}:} 
	
	۱- با کلیک بر روی قسمت نام کاربری و انتخاب گزینه \lr{Settings} کاربر به صفحه ویرایش اطلاعات کاربری هدایت می‌شود.
	
	۲- کاربر می‌تواند رمز عبور، ایمیل و تلفن همراه خود را تغییر دهد.
	
	۳- پس از انجام تغییرات، برای اطمینان از هویت کاربر یک کد برای کاربر پیامک می‌شود و پس از این که کاربر آن را وارد کرد، اطلاعات تغییر خواهند کرد.
	
	\textbf{نیازمندی‌های \lr{Non-Functional}:}
	
	
	۱- روال ارسال پیامک باید کوتاه باشد.
	
	۲- رمز عبور نباید به صورت \lr{Plain Text} نمایش داده شود.
	
	۳- صفحه نغییر اطلاعات باید ساده و قابل فهم طراحی شده باشد.
	
	۴- در صورت موفقیت یا شکست خوردن عملیات، رنگ نوشته‌ها باید مطابق انتظار باشد. (مثلا سبز برای موفقیت و قرمز برای خطا)
	
	۵- روال تغییر اطلاعات باید امن و فارغ از هرگونه نشت اطلاعات حساس کاربری باشد.
	
	۶- روال تغییر اطلاعات باید در برابر حملاتی مانند \lr{SQL Injection} مقاوم باشد.
	
	\textbf{ارتباط سناریو با نیازمندی‌ها:}
	
	در این سناریو،‌ کاربر اطلاعات کاربری خود را ویرایش می‌کند. از آن‌جا که این اطلاعات برای ورود مجدد به سایت حیاتی هستند،‌ لازم است احراز هویت مجدد با استفاده از پیامک نیز انجام شود تا اطمینان حاصل کنیم اطلاعات کاربری توسط فردی به غیر از صاحب حساب تغییر داده نمی‌شوند.
	
	
	\subsection{\LARGE{سناریو: بازیابی رمز عبور فراموش شده}}
	
	\textbf{انجام‌دهنده:}
	
	کاربری که قصد ورود به سایت را دارد و رمز عبور خود را به یاد نمی‌آورد.
	
	\textbf{هدف:}
	
	کاربر می‌تواند یک رمز عبور جدید برای ورود به سایت بسازد و وارد سایت شود.
	
	\textbf{نیازمندی‌های \lr{Functioanl}:} 
	
	۱- در صفحه ورود به سایت باید دکمه «رمز عبورم را فراموش کرده‌ام» وجود داشته باشد.
	
	۲- با کلیک روی دکمه مذکور، کاربر باید ایمیل یا تلفن همراه خود را وارد کند.
	
	۳- یک لینک یکبار مصرف برای کاربر ارسال شود.
	
	۴- با کلیک بر روی لینک، کاربر به صفحه‌ای هدایت می‌شود و در آن جا رمز عبور جدید خود را وارد می‌کند.
	
	۵- پس از تایید، کاربر باید بتواند با رمز عبور جدیدی که تنظیم کرده است وارد سایت شود.
	
	\textbf{نیازمندی‌های \lr{Non-Functional}:}
	
	۱- روال ارسال پیامک باید کوتاه باشد.
	
	۲- رمز عبور نباید به صورت \lr{Plain Text} نمایش داده شود.
	
	۳- در صورت موفقیت یا شکست خوردن عملیات، رنگ نوشته‌ها باید مطابق انتظار باشد. (مثلا سبز برای موفقیت و قرمز برای خطا)
	
	۵- روال بازیابی رمز عبور باید امن و فارغ از هرگونه نشت اطلاعات حساس کاربری باشد.
	
	۶- روال بازیابی رمز عبور باید در برابر حملاتی مانند \lr{SQL Injection} مقاوم باشد.
	
	۷- اطمینان حاصل شود که لینگ ارسال شده به کاربر، یک بار مصرف است.
	
	\textbf{ارتباط سناریو با نیازمندی‌ها:}
	
	
	در این سناریو کاربر می‌تواند یک رمز عبور جدید برای ورود به سایت تنظیم کند. لازم این کار با استفاده از لینک یکبار مصرف انجام شود که از تغییرپذیری رمز ورود در صورت لو رفتن لینک جلوگیری به عمل آید.
	
	
	\subsection{\LARGE{سناریو: ساخت روزمه جدید}}
	
	\textbf{انجام‌دهنده:}
	
	کاربر وارد شده به سایت
	
	\textbf{هدف:}
	
	کاربر می‌تواند یک روزمه جدید برای خود بسازد و اطلاعات خود را درون آن وارد کند
	
	\textbf{نیازمندی‌های \lr{Functioanl}:} 
	
	۱- با کلیک روی دکمه ساخت روزمه جدید کاربر باید به صفحه ساخت روزمه جدید هدایت شود.
	
	۲- کاربر می‌تواند یکی از قالب‌های آماده برای ساخت روزمه را انتخاب کند.
	
	۳- اگر رزومه در حال ساخت، اولین روزمه‌ای است که فرد آن را می‌سازد، این روزمه باید به صورت خودکار به عنوان رزومه‌ای از کاربر که نمایش داده‌ خواهد شد انتخاب شود.
	
	۴- کاربر به ترتیب باید اطلاعات خود مانند نام و نام خانوادگی، سن، توصیحاتی در مورد خود، مهارت‌های خود و میزان سابقه خود در آن‌ها، سوابق شغلی و تحصیلی، مهارت‌های نرم، گواهینامه‌ها، پروژه‌ها، نمرات و راه‌های ارتباطی خود را در فیلد‌های مربوط به خودشان وارد کند. البته وارد کردن این اطلاعات اختیاری است.
	
	۵- در هر مرحله کاربر باید بتواند اطلاعات وارد شده را ذخیره کند.
	
	۶- پس از اتمام فرآیند باید اطلاعات کاربر درون پایگاه داده ذخیره شود.
	
	۷- پس از پایان ساخت رزومه، کاربر باید بتواند از رزومه خود خروجی \lr{PDF} بگیرد یا لینک صفحه‌ای که رزومه وی را نمایش می‌دهد را کپی کند و با دیگران به اشتراک بگذارد.
	
	۸- کاربر باید بتواند لینک روزمه خود را به همراه یک متن در شبکه‌های اجتماعی منتشر کند.
	
	\textbf{نیازمندی‌های \lr{Non-Functional}:}
	
	۱- فرم دریافت اطلاعات نباید شلوغ و گیج‌کننده باشد. بلکه اطلاعات کاربری به تفکیک گفته شده در بالا، در طی چند مرحله از کاربر دریافت شود.
	
	۲- در صورت موفقیت یا شکست خوردن عملیات ثبت‌نام رنگ نوشته‌ها باید مطابق انتظار باشد. (مثلا سبز برای موفقیت و قرمز برای خطا)
	
	۳- روال ثبت‌نام باید امن و فارغ از هرگونه نشت اطلاعات حساس کاربری باشد.
	
	۴- روال ثبت‌نام باید در برابر حملاتی مانند \lr{SQL Injection} مقاوم باشد.
	
	۵- اطلاعات وارد شده در روزمه نباید شامل فحاشی، الفاظ رکیک و توهین به هویت ملی و اعتقادات و سنت‌های یک قوم یا مذهب باشد.
	
	\textbf{ارتباط سناریو با نیازمندی‌ها:}
	
	کاربر می‌تواند با انتخاب یک قالب و وارد کردن اطلاعات خود یک رزومه رسمی و حرفه‌ای داشته باشد. این کار به افرادی که نمی‌دانند در رزومه خود چه چیزهایی بنویسند کمک می‌کند که با وارد کردن اطلاعات مورد نیاز اجزای یک رزومه حرفه‌ای را شکل دهند. همچنین برای پیشگیری از نوشتن مطالب خارج از عرف و فحاشی یک موتور هوش مصنوعی باید اطلاعات را بررسی کند و در صورت وجود کلمات حساس از ایجاد شدن رزومه جلوگیری کند. امکان دانلود رزومه به صورت \lr{PDF} برای ارائه به کارفرمایان دیگر نیز وجود دارد.
	
	
	\subsection{\LARGE{سناریو: تغییر دادن یک روزمه}}
	
	\textbf{انجام‌دهنده:}
	
	کاربر وارد شده به سایت
	
	هدف: کاربر می‌تواند یکی از روزمه‌های خود را که پیشتر ساخته است تغییر دهد و اطلاعات آن را ویرایش کند یا اینکه آن را حذف کند.
	
	\textbf{نیازمندی‌های \lr{Functioanl}:} 
	
	۱- کاربر باید بتواند از میان روزمه‌های خود یکی را انتخاب و آن را ویرایش کند.
	
	۲- با کلیک روی گزینه ویرایش، کاربر باید بتواند اطلاعاتی که در زمان ساخت روزمه وارد کرده است را به نوبت تغییر دهد و آن را ذخیره کند. همچنین کاربر باید بتواند قالب رزومه خود را عوض کند.
	
	۳- کاربر باید بتواند یکی از رزومه‌ها را به عنوان رزومه‌ای که در صفحه وی نمایش خواهد شد انتخاب کند.
	
	۴- کاربر باید بتواند یک رزومه را حذف کند.
	
	\textbf{نیازمندی‌های \lr{Non-Functional}:}
	
	۱- فرم تغییر رزومه نباید شلوغ و گیج‌کننده باشد. بلکه اطلاعات کاربری به تفکیک گفته شده در بالا، در طی چند مرحله از کاربر دریافت شود.
	
	۲- در صورت موفقیت یا شکست خوردن عملیات ثبت‌نام رنگ نوشته‌ها باید مطابق انتظار باشد. (مثلا سبز برای موفقیت و قرمز برای خطا)
	
	۳- روال تغییر رزومه باید امن و فارغ از هرگونه نشت اطلاعات حساس کاربری باشد.
	
	۴- روال تغییر رزومه در برابر حملاتی مانند \lr{SQL Injection} مقاوم باشد.
	
	
	\textbf{ارتباط سناریو با نیازمندی‌ها:}
	
	در صورت نیاز، کاربر می‌تواند هر یک از اجرای روزمه را عوض کند و یک روزمه را به عنوان رزومه اصلی خود انتخاب کند.
	
	
	\subsection{\LARGE{سناریو: مشاهده روزمه کاربری}}
	
	\textbf{انجام‌دهنده:}
	
	هر کاربر
	
	\textbf{هدف:}
	
	با کلیک روی لینک رزومه یک فرد، کاربر باید بتواند روزمه وی را مشاهده کند
	
	\textbf{نیازمندی‌های \lr{Functioanl}:} 
	
	۱- با کلیک روی لینک روزمه، کاربر باید به صفحه فرد صاحب رزومه هدایت شود.
	
	۲- اطلاعاتی از فرد صاحب رزومه که توسط وی در رزومه ثبت شده است باید نمایش داده شود.
	
	۳- در صورتی که روزمه‌ای متناظر با لینک داده شده وجود نداشته باشد، باید خطای ۴۰۴ بازگردانده شود
	
	\textbf{نیازمندی‌های \lr{Non-Functional}:}
	
	۱- در صورت موفقیت یا شکست خوردن عملیات ثبت‌نام رنگ نوشته‌ها باید مطابق انتظار باشد. (مثلا سبز برای موفقیت و قرمز برای خطا)
	
	۲- روال ثبت‌نام باید امن و فارغ از هرگونه نشت اطلاعات حساس کاربری باشد.
	
	۳- روال ثبت‌نام باید در برابر حملاتی مانند \lr{SQL Injection} مقاوم باشد.
	
	۴- اطلاعات کاربری باید مطابق با قالب انتخاب شده نمایش داده شوند.
	
	۵- اطمینان حاصل شود که لینک مورد نظر به صفحه‌ی درستی هدایت می‌شود و رزومه فرد دیگری نمایش داده نمی‌شود.
	
	
	\textbf{
		ارتباط سناریو با نیازمندی‌ها:}
	
	این سناریو امکان مشاهده یک رزومه در سایت را فراهم می‌سازد. در واقع این جا جاییست که فرد رزومه خود در معرض دید همگان قرار می‌دهد.
	
	\subsection{\LARGE{سناریو: جستجو در سایت}}
	
	\textbf{انجام‌دهنده:}
	
	هر کاربر
	
	\textbf{هدف:}
	
	می‌توان با جستجو در سایت افراد با ویزگی‌های خاصب را فیلتر کرد و روزمه آن‌ها را مشاهده نمود.
	
	\textbf{نیازمندی‌های \lr{Functioanl}:} 
	
	۱- در صفحه اصلی کاربر باید بتواند با فیلتر کردن گزینه‌هایی مانند شهر محل سکونت، مهارت مورد نظر خود، سطح تحصیلات مورد نظر، سابقه کاری فرد و نام فرد یک جستجو در سایت انجام دهد.
	
	۲- تمامی رکورد‌های حاوی اطلاعاتی درخواستی جستجو باید نمایش داده شوند.
	
	۳- کاربر باید بتواند با کلیک بر روی هر رکورد،‌ رزومه هر فرد را مشاهده کند.
	
	۴- نتیجه جستجو باید در قالب کارت و در دسته‌های ۱۲ تایی نمایش داده شوند و \lr{Paging} هم انجام شود.
	
	۵- در صورت پیدا نشدن روزمه‌ای با فیلترهای اعمال شده باید پیغام مناسب نمایش داده شود.
	
	\textbf{نیازمندی‌های \lr{Non-Functional}:}
	
	۱- در صورت موفقیت یا شکست خوردن عملیات ثبت‌نام رنگ نوشته‌ها باید مطابق انتظار باشد. (مثلا سبز برای موفقیت و قرمز برای خطا)
	
	۲- روال جستجو باید در برابر حملاتی مانند \lr{SQL Injection} مقاوم باشد.
	
	\textbf{ارتباط سناریو با نیازمندی‌ها:}
	
	کارفرمایان و افرادی که دنبالی افرادی با مهارت‌های خاصی می‌گردند می‌توانند با اعمال فیلترهای موجود نیروی کار مورد نیاز خود را پیدا کنند.
	
	
	\subsection{\LARGE{سناریو: نوشتن توصیه نامه}}
	
	\textbf{انجام‌دهنده:}
	
	کاربر وارد شده در سایت
	
	\textbf{هدف:}
	
	یک کاربر می‌تواند روی رزومه فرد دیگری توصیه‌نامه بنویسد و این توصیه‌، با تایید صاحب رزومه در رزومه وی نمایش داده خواهد شد.
	
	\textbf{نیازمندی‌های \lr{Functioanl}:} 
	
	۱- در صفحه نمایش یک رزومه، با کلیک روی گزینه نوشتن توصیه‌نامه، کاربر به صفحه درج توصیه‌نامه هدایت می‌شود
	
	۲- در این صفحه توضیحات کاربر به همراه اسناد مربوطه (اختیاری) دریاقت می‌شود.
	
	۳- یک نوتیفیکیشن برای صاحب رزومه ارسال می‌شود که با تایید یا رد کردن آن می‌تواند توصیه‌نامه را درون روزمه خود درج کند.
	
	۴- توصیه‌نامه درج شده در در قسمت «توصیه‌‌نامه‌های نوشته شده» در رزومه فرد توصیه‌کننده نیز درج می‌شود. 
	
	
	\textbf{نیازمندی‌های \lr{Non-Functional}:}
	
	۱- روال درج توصیه نامه باید ساده و کوتاه باشد
	
	۲- پیغام مناسب در صورت مواجه شدن با خطا یا موفقیت با رنگبندی مناسب نمایش داده شود.
	
	۳- روال نوشتن توصیه نامه باید در برابر حملاتی مانند \lr{SQL Injection} مقاوم باشد.
	
	۴- روال نوشتن توصیه نامه باید امن و فارغ از هرگونه نشت اطلاعات حساس کاربری باشد.
	
	\textbf{
		ارتباط سناریو با نیازمندی‌ها:}
	
	یک کاربر می‌تواند با صرف کردن اعتبار کاری خود برای یک فرد دیگر توصیه نامه بنویسد. درج شده توصیه نوشته شده در رزومه خود فرد یک مزیت دارد. در واقع نمایش داده می‌شود که فرد اعتبار کاری خود را برای چه کسانی هزینه کرده است. این کار از نوشتن توصیه‌نامه‌های بیهوده و تعریف و تمجید‌های دوستانه پیشگیری می‌کند.
	
	
	\subsection{\LARGE{\textbf{\lr{HW5}}:}}
	\subsubsection{\Large{سناریو ثبت نام در سایت}}
	\textbf{\lr{WHO}:} کاربری جدیدی که پیشتر در سایت ثبت‌نام نکرده و حساب کاربری ندارد و قصد دارد با ثبت‌نام در سایت از امکانات برنامه رزومه‌ساز استفاده کند.
	
	\textbf{\lr{WHAT}:} دنبال چه چیزی هستیم؟ دنبال این هستیم که کاربر بتواند با طی کردن یک روال قابل‌فهم و ساده، در چند گام در سایت ثبت‌نام کند.
	
	\textbf{\lr{WHY}:} چرا کاربر باید ثبت‌نام کند؟ برای ساخت رزومه و یا نوشتن توصیه‌نامه روی رزومه‌های دیگر کاربر باید در سایت ثبت‌نام کرده باشد تا بتواند از قابلیت‌های آن استفاده کند.
	
	\textbf{\lr{HOW}:} با کلیک بر روی گزینه ثبت‌نام در هدر بالای صفحه کاربر به صفحه ثبت‌نام هدایت می‌شود. با وارد کردن آدرس ایمیل و یک رمز عبور در سایت ثبت‌نام می‌کند. سپس باید با استفاده از لینکی که به ایمیل او فرستاده می‌شود ثبت‌نام خود را تکمیل کند. همچنین لازم است امکان ثبت‌نام با اکانت گوگل هم فراهم باشد.
	
	\textbf{\lr{WHEN}:} این قابلیت باید به صورت مختصر،‌یعنی ثبت‌نام با ایمیل و رمز عبور در نسخه اولیه وجود داشته باشد. در نسخه‌های بعدی هم صحت‌سنجی ایمیل و بعدتر ثبت‌نام با گوگل اضافه شود.
	
	
	\subsubsection{\Large{سناریو نوشتن توصیه‌نامه}}
	\textbf{\lr{WHO}:} کاربری که پیش‌تر در سایت ثبت‌نام کرده و قصد دارد روی رزومه‌ی یکی از افرادی که می‌شناسد، توصیه‌نامه‌ای را بنویسد
	
	\textbf{\lr{WHAT}:} دنبال این هستیم که یک کاربر بتواند توصیه‌نامه‌ای را رزومه دیگری بنویسد و صاحب رزومه در صورت تمایل و تایید آن، توصیه‌نامه را به روزمه‌اش اضافه کند. همچنین توصیه‌نامه نوشته شده در روزمه فرد نویسنده هم اضافه خواهد شد.
	
	\textbf{\lr{WHY}:} این قابلیت مي‌تواند جایگاه رزومه افراد را بالاتر ببرد. در واقع فردی که توصیه‌نامه دریافت می‌کند، یک تایید قطعی و قوی از افراد مرتبط با خود دریافت می‌کند و با نوشتن توصیه‌نامه نیز اعتبار کاری خود را برای دیگران خرج می‌کند.
	
	\textbf{\lr{HOW}:} یک کاربر که درسایت ثبت‌نام کرده و وارد حساب خود شده است، با مشاهده رزومه فرد مورد نظر خود میتواند  روی گزینه نوشتن توصیه‌نامه کلیک کند و به صفحه نوشتن توصیه‌نامه هدایت شود. در آن‌جا متن مورد نظر خود را می‌نویسد و روی دکمه تایید کلیک می‌کند. پس از این کار، یک نوتیفیکشن برای فرد صاحب رزومه ارسال می‌شود تا توصیه‌نامه را مشاهده کند. در صورت تایید صاحب‌ رزومه، این توصیه‌نامه به رزومه فرد افزوده می‌شود. همچنین توصیه‌ نامه به رزومه فرد نویسنده هم اضافه می شود تا مشخص شود فرد، برای چه کسانی از اهتبار خود مایه گذاشته است.
	
	\textbf{\lr{WHEN}:} این قابلیت، قابلیت ضروری‌ای برای محصول ما نیست. پس از انتشار نسخه‌ اولیه و همچنین لانچ شدن پروژه به عنوان یک فیچر جدید می‌توانیم به آن نگاه کنیم و آن را به برنامه اضافه کنیم.
	
	
	
	\section{\huge{اعتبارسنجی نیازمندی‌ها}}
	
	اعتبارسنجی نیازمندی‌ها را برای سناریوی «ثبت‌نام کاربران» بررسی می‌کنیم:
	
	برای این موضوع به بررسی پاسخ بعضی از سوالات چک لیست اعتبارسنجی نیازمندی‌ها از کتاب مهندسی نرم افزار
	\lr{Roger S. Pressman}\LTRfootnote{Software Engineering A Practitioners Approach 9th Edition}
	استفاده می‌کنیم. (این چک لیست سوالات در صفحۀ 106 کتاب قرار دارد.) 
	
	در این سناریو خواستۀ کاربر که ثبت نام و ایجاد یک حساب کاربری در سایت است به وضوح شرح داده شد و با توجه به اعلام نیازمندی‌های کارکردی و غیرکارکردی برای آن گام به گام مشخص است که باید چگونه قدم برداشت. همچنین ذینفعان آن خواسته، مقررات مورد نیاز و... شرح شده است. (این مقررات بخشی از نیازمندی‌های کارکردی و غیرکارکردی می‌باشند.) 
	نیازمندی فوق می‌تواند به نیازمندی‌ «ورود کاربران به سایت» ارتباط داشته باشد. چون در صفحۀ ورود کاربران به سایت ما بخش ثبت نام کاربران را خواهیم داشت. همچنین مشخص است که نیامندی فوق تاثیری بر محدودیت‌های دامنۀ سیستم نخواهد داشت. 
	سناریوی ثبت‌نام کاربران نیازمند آزمایش‌های همه جانبه است، از ذخیرۀ اطلاعات کاربر و ایجاد حساب کاربری برای او گرفته تا تضمین حفظ و ایمنی داده‌های کاربر شامل می‌شود. همینطور می‌توانیم برای بررسی این آزمایش‌ها معیار تعیین کنیم.
	با توجه به استفاده از وب سرویس برای ایجاد سامانۀ رزومه ساز، ثبت‌نام کاربران از طریق هر 
	سیستمی که به وب دسترسی داشته باشد ممکن خواهد بود.
	ثبت نام کاربران به گونه ای ساختار دهی شده است که به راحتی قابل درک می‌باشد و به محصول کاری بهتری منجر می‌شود‌.
	برای ساختار دهی ثبت نام کاربران ورودی های مورد نیاز عبارت است از یک اکانت جمیل معتبر ، یا اسم و شماره تلفن او، که هنگام ثبت نام وارد می‌کند.
	و این ورودی ها بر حسب نوع آن ها  ذخیره خواهند شد، به طور مثال ذخیره سازی ایمیل های کاربران در ساختاری جدا از شماره تلفن آن ها خواهد بود.
	همچنین این مشخصه در سند نیازمندی دارای جزئیاتی کافی می‌باشد تا توسعه‌دهندگان بتوانند آن را درک کنند.
	
	همچنین طرحی برای توسعۀ ثبت نام کاربران، طراحی این نیازمندی در ساخت محصول نهایی، طراحی باکس‌ها و گزینه‌های  مورد نظر،  به عنوان شاخصی برای مشخصات نیازمندی فوق در نظر گرفته شده می‌شود.
	همچنین نیازمندی‌های مرتبط با عملکرد سیستم و ویژگی های عملیاتی آن، که نظیر 
	رابط کاربری ساده و کارآمد، تبدیل فرمت فایل روزمه، وارد کردن تجربۀ کاری و مهارت‌ها و ....می‌باشد، از نیازمندی فوق جدا شده است. 
	همچنین این نیازهای عملیاتی که در عبارت فوق ذکر شد جزو نیازهای ضمنی است به گونه‌ای که مشتری هنگام کار با سیستم انتظار وجود آن‌ها را دارد.
	
	
	\section{\huge{تحلیل ریشه‌ها}}
	برای اعمال RCA دو مورد از مشکلات احتمالی را در نظر گرفته‌ایم و با استفاده از روش ۵ چرا سعی کرده‌ایم عوامل اصلی مشکل را پیدا کنیم.
	
	در سناریوی بازیابی رمز عبور برای مشکل زیر می‌توانیم ۵ سوال دنباله‌دار را مطرح کنیم.
	
	مشکل: ممکن است گاهی اوقات کاربران لینک بازیابی رمز عبور را دریافت کنند.
	
	چرای اول: چرا برخی از کاربران لینک بازیابی رمز عبور را دریافت نمی‌کنند؟
	
	پاسخ: ممکن است ایمیل یا شماره تلفن خود را به درستی وارد نکرده باشند.
	
	چرای دوم: چرا کاربران ایمیل یا شماره تلفن را اشتباه وارد می‌کنند؟
	
	پاسخ: احتمالا کاربران فراموش می‌کنند که از کدام ایمیل یا شماره تلفن خود برای ثبت‌نام استفاده کرده‌اند.
	
	چرای سوم: چرا کاربران ایمیل یا شماره تلفن مورد استفاده خود را فراموش می‌کنند؟
	
	پاسخ: هیچ آپشنی برای یادآوری یا اشاره به نسیت قبلی کابران در صفحه ورود وجود ندارد.
	
	چرای چهارم: چرا آپشن نگهداری اطلاعات نشست وجود ندارد؟
	
	پاسخ: طراحی صفحه ورود به سیستم شامل ویژگی یادآور یا اشاره نیست.
	
	چرای پنچم: چرا حفظ نشست قبلی در نسخه گنجانده نشده است؟
	
	پاسخ: به چنین موضوعی توجه نشده و اصلا هم در اولویت نبوده است.
	
	
	در سناریوی وارد کردن اطلاعات رزومه نیز یک مشکل را بررسی کرده‌ایم.
	
	مشکل: در برخی از موارد کاربران نمی‌توانند اطلاعات رزومه خود را با موفقیت ذخیره کنند.
	
	چرای اول: چرا کاربران در ذخیره اطلاعات رزومه خود مشکل دارند؟
	
	پاسخ: با فشردن دکمه «ذخیره» پاسخی دریافت نمی‌شود.
	
	چرای دوم: چرا دکمه‌ی ذخیره کار نمی‌کند؟
	
	پاسخ: سرور درخواست ذخیره را به درستی پردازش نمی‌کند.
	
	چرای سوم: چرا سرور درخواست ذخیره را به درستی پردازش نمی‌کند؟
	
	پاسخ: سرور با ترافیک بالایی مواجه است که باعث تاخیر در پردازش درخواست‌ها می‌شود.
	
	چرای چهارم: چرا باز زیادی روی سرور قرار دارد دارد؟
	
	پاسخ: زیرساخت اختصاص داده‌شده به سرور برای مدیریت حجم تعداد درخواست کاربران فعلی کافی نیست.
	
	چرای پنجم: چرا زیرساخت موجود برای مدیریت حجم درخواست‌های فعلی کفایت نمی‌کند؟
	
	پاسخ: در مرحله طراحی، تخمین داده شده برای تعداد کاربران در برهه فعلی تخمین درستی نبوده است.
	
	\section{\huge{معماری مفهومی برای سیستم رزومه ساز}}
	\begin{enumerate}
	\item 
	لایۀ {Presentation}\lr:
	\subitem
	\lr{:Resume Builder Interface -}
	\subsubitem
	نمایش فرم آنلاین برای وارد کردن اطلاعات شخصی و حرفه‌ای.
	\subsubitem
	فراهم کردن امکان انتخاب قالب رزومه.
	\subsubitem
	نمایش پیش‌نمایش فوری رزومه با تغییرات.
	\item 
	لایۀ {Application Logic}\lr:
	\subitem
	\lr{:User Profile Management -}
	\subsubitem
	ذخیره و مدیریت اطلاعات کاربران شامل نام، تحصیلات، تجربیات و مهارت‌ها.
	\subitem
	\lr{:Resume Generation Engine -}
	\subsubitem
	تولید رزومه با توجه به اطلاعات وارد شده.
	\subsubitem
	امکان انتخاب بخش‌ها و ترتیب آنها در رزومه.
	\item 
	لایۀ \lr{Business Logic}:
	\subitem
	\lr{:Skill Matching Algorithm -}
	\subsubitem
	الگوریتمی برای ارتباط مهارت‌های کاربر با نیازهای شغلی.
	\subitem
	\lr{:Dynamic Content Suggestions -}
	\subsubitem
	ارائه پیشنهادات بر اساس نوع شغل و حوزه فعالیت.
	\subitem
	\lr{:Resume Customization Recommendations -}
	\subsubitem
	اطلاعات و پیشنهادات برای بهبود رزومه کاربر.
	
	\item 
	لایۀ \lr{Data Management}:
	\subitem
	\lr{:User Profiles Database -}
	\subsubitem
	ذخیره اطلاعات کاربران در پایگاه داده.
	\subitem
	\lr{:Resume Templates Database -}
	\subsubitem
	ذخیره و مدیریت قالب‌های رزومه مختلف.
	\subitem
	\lr{:Job Descriptions Database -}
	\subsubitem
	ذخیره اطلاعات شغلی و نیازمندی‌ها.
	
	\item 
	لایۀ \lr{Integration}:
	\subitem
	\lr{:Third-Party APIs -}
	\subsubitem
	اتصال به سرویس‌های استخدامی برای به‌روزرسانی نیازمندی‌های شغلی.
	\subsubitem
	اشتراک گذاری خودکار رزومه در شبکه‌های اجتماعی.
	\item
	لایۀ \lr{Security}:
	\subitem
	\lr{:User Authentication and Authorization -}
	\subsubitem
	اطمینان از امنیت ورود و اعتبارسنجی کاربران.
	\subsubitem
	مدیریت دسترسی‌ها و مجوزهای مرتبط با داده‌ها و عملکردهای سیستم.
	\subitem
	\lr{:Data Encryption -}
	\subsubitem
	رمزنگاری اطلاعات حساس کاربران در پایگاه داده و در ارتباط با سرویس‌های خارجی.
	\subitem
	\lr{:Secure API Communication -}
	
	\end{enumerate}
	
	\subsection{\LARGE{توضیحات}}
	\begin{enumerate}
	\item 
	 کاربران اطلاعات خود را در واسط کاربری وارد کرده و با استفاده از المان‌های 
	 \lr{Presentation}، رزومه خود را ساخته و نمایش می‌دهند.
	
	\item 
	لایه
	 \lr{Application Logic} 
	 شامل منطق مربوط به مدیریت کاربران و تولید رزومه است.
	
	\item 
	 \lr{Business Logic} 
	 اقدام به پیشنهاد و بهبود محتوای رزومه بر اساس تجربیات و مهارت‌های کاربر میکند.
	
	\item
	 \lr{Data Management} 
	لایه‌ای است که اطلاعات کاربران، قالب‌های رزومه، و نیازمندی‌های شغلی را مدیریت می‌کند.
	
	\item
	 لایه 
	 \lr{Integration}
	 اتصالات به سرویس‌های خارجی را فراهم می‌کند تا اطلاعات نیازمندی‌های شغلی به‌روز شوند و رزومه به شبکه‌های اجتماعی ارسال شود.
	
	\item
	 لایه
	  \lr{Security}
	 اضافه شده به معماری برای افزایش امنیت سیستم.
	
	\item
	 \lr{Authentication} و \lr{Authorization} 
	مسئول اعتبارسنجی کاربران و مدیریت دسترسی‌هاست.
	
	\item
	 \lr{Data Encryption} 
	اطلاعات حساس در پایگاه داده و ارتباط با سرویس‌های خارجی را رمزنگاری می‌کند.
	
	\item
	\lr{Secure API Communication} 
	از امنیت ارتباطات با سرویس‌های خارجی اطمینان حاصل می‌کند.
	
	\end{enumerate}
	
	\subsection{\LARGE{ارتباطات}}
	\begin{enumerate}
	\item 
	 ارتباط
	  \lr{Presentation}
	   و
	   \lr{Application Logic}:
	\subitem
	   - واسط کاربری
	    (\lr{Presentation}) 
	    توسط کاربر برای وارد کردن اطلاعات استفاده می‌شود.
	\subitem
	   \lr{Application Logic} مسئول تجزیه و تحلیل اطلاعات وارد شده توسط کاربر و ایجاد رزومه است.
	
	\item 
	 ارتباط
	  \lr{Application Logic} و 
	  \lr{Business Logic}:
	
	\subitem
	   \lr{Application Logic} 
	   بر اساس اطلاعات کاربران، به
	    \lr{Business Logic} 
	    نیازها و مهارت‌های مرتبط با شغل را ارسال می‌کند.
	\subitem
	    \lr{Business Logic} از این اطلاعات برای ارائه پیشنهادات بهبود رزومه و انجام الگوریتم‌های مرتبط با مهارت‌ها استفاده می‌کند.
	
	\item 
	ارتباط
	 \lr{Business Logic} 
	 و
	 \lr{Data Management}:
	
	\subitem
	 \lr{Business Logic} نیاز به دسترسی به اطلاعات کاربران، قالب‌های رزومه، و اطلاعات مرتبط با نیازمندی‌های شغلی دارد.
	\subitem
	 \lr{Data Management} این اطلاعات را از پایگاه داده استخراج کرده و به 
	 \lr{Business Logic} ارائه می‌دهد.
	\item 
	ارتباط 
	\lr{Data Management} و \lr{Integration}:
	
	\subitem
	    \lr{Data Management} 
	    اطلاعات را از پایگاه داده مدیریت می‌کند و به
	     \lr{Integration} ارائه می‌دهد.
	\subitem
	   \lr{Integration} از این اطلاعات برای اتصال به سرویس‌های خارجی (مانند سرویس‌های استخدامی) استفاده می‌کند.
	
	\item 
	ارتباط 
	\lr{Integration} و \lr{Presentation}:
	\subitem
	   \lr{Integration} ممکن است اطلاعات به‌روزرسانی شده از سرویس‌های خارجی را به 
	   \lr{Presentation} ارسال کند تا به کاربر نمایش داده شود.
	\subitem
	   \lr{Presentation} ممکن است امکانات اشتراک‌گذاری رزومه در شبکه‌های اجتماعی را از طریق \lr{Integration} فراهم کند.
	
	\item 
	\lr{Security} به همه لایه‌ها:
	\subitem
	\lr{Security} لایه‌ای است که به تمام لایه‌ها ارتباط دارد. به‌عنوان مثال، اطلاعات ارسالی از \lr{Presentation} به \lr{Application Logic} از طریق لایه \lr{Security} رمزنگاری می‌شود تا امنیت اطلاعات اطمینان حاصل شود. همچنین، دسترسی‌ها و مجوزها نیز تحت کنترل لایه \lr{Security} قرار دارند.
	
	\end{enumerate}
	
	این ارتباطات نشان‌دهنده جریان داده و کنترل بین اجزاء مختلف سیستم رزومه ساز می‌باشد.
	
	
	
	
	
	
	
	
	
	
	
	
	
	
	
\end{document}




