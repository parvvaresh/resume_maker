\documentclass[]{article}
\usepackage{color}
\usepackage{graphicx}
\usepackage{fancyhdr}
\usepackage{geometry}
\usepackage{verbatim}
\usepackage[pagebackref=true,colorlinks,linkcolor=blue,citecolor=green]{hyperref}
\usepackage{ptext}
\usepackage{atbegshi}
\usepackage{xepersian}
\settextfont{Zar}
\settextfont{XB Niloofar}
\setlatintextfont{Junicode}

\usepackage[width=0.00cm, height=0.00cm, left=2cm, right=1.5cm, top=2.50cm, bottom=2.5cm]{}

\title{
	\begin{center}
		\includegraphics[width=8cm,height=10cm]{AUT.png}
	\end{center}	
	
	\LARGE
	درس: اصول طراحی نرم افزار \\
	\Large
	استاد: دکتر احسان علیرضایی \\
	\Large
	فاز یک پروژه
	
}
\author{
	گروه 4: \\
	مهدی مهدوی - 9833067 \\
	علیرضا پرورش - 9912013 \\
	آروین اسدی - 9913701 \\
	امیرمحمد کمانی - 9913704
}
\fancypagestyle{logo}{
	\fancyhf{}
	\fancyhfoffset[R]{3.4cm}\fancyfoot[CE,CO]{~\\[.5cm]\thepage}
	\fancyhfoffset[L]{3.4cm}
	\fancyhead[RE,RO]{~\\[-2cm]\includegraphics[height=2.3cm]{MCS}}
	\renewcommand{\headrulewidth}{0pt}
	
}
\pagestyle{logo}

\textheight=21.5cm

\begin{document}
	\maketitle
	\thispagestyle{empty}
	\newpage
	\tableofcontents
	\newpage
	
	\fontsize{12pt}{14pt}
	\LARGE
	%	\selectfont
	
	\section{\huge{\lr{Diagrams in Enterprise Architect}}}
		دیاگرام‌های خواسته شده در این نرم افزار را می‌توانید در فایل 
		
 \lr{P1\_9833067\_9912013\_9913701\_9913704.qea}
		مشاهده کنید. لازم به ذکر است فایل فوق حاوی 1 
		\lr{Requirement Diagram}،
		9 تا 
		\lr{Use Case}
		(\lr{use case} 
		سناریوهای 2 و 5 یکسان است و نام گذاری سایر 
		\lr{use case}ها
		متناسب با شمارۀ سناریوهاست.)، 9 تا 
		\lr{Activity Diagram}،
		2 تا 
		\lr{Sequence Diagram} و 
		1 \lr{Class Diagram} 
		می‌باشد.
		
	\section{\huge{تحلیل ریشه‌ای}}
	برای تحلیل علل ریشه‌ای مسئله طراحی اپلیکیشن رزومه ساز، می‌توان از روش‌های مختلف تحلیل علت ریشه‌ای استفاده کرد.
	ابتدا، یکی از روش‌های معروف آن یعنی "۵ چرا" را بررسی میکنیم:
	روش ۵ چرا یا همان
	\lr{The 5 whys}
	
	مشکل:
	
	کیفیت نامناسب رابط کاربری و تجربه کاربری در اپلیکیشن.
	
	چرا؟ (شماره ۱)
	
	چرا رابط کاربری و تجربه کاربری ضعیف است؟
	
	دلیل: زیرا نیازهای واقعی کاربران در نظر گرفته نشده است.
	
	چرا؟ (شماره ۲)
	
	چرا نیازهای واقعی کاربران در نظر گرفته نشده است؟
	
	دلیل: چون که نقدها و بازخوردهای کاربران به درستی مورد بررسی قرار نگرفته است.
	
	چرا؟ (شماره ۳)
	
	چرا نقدها و بازخوردهای کاربران به درستی مورد بررسی قرار نگرفته است؟
	
	دلیل: چون فرآیند جامعی برای ارزیابی و بازخورد کاربران وجود ندارد!
	
	چرا؟ (شماره ۴)
	
	چرا فرآیند جامعی برای ارزیابی و بازخورد کاربران وجود ندارد؟
	
	دلیل: به علت عدم وجود یک سیستم سازمانی یا فرآیند مشخص برای جمع آوری و ارزیابی بازخوردهای کاربران‌.
	
	چرا؟ (شماره ۵)
	
	چرا یک سیستم سازمانی یا فرآیند مشخصی برای جمع آوری و ارزیابی بازخوردهای کاربران‌ وجود ندارد؟
	
	دلیل: به علت عدم تخصیص منابع یا عدم توجه کافی به نیازمندی‌های کاربران در فرآیند توسعه و طراحی.
	
	همان گونه که در ابتدا اشاره شد، روش‌های دیگری برای تحلیل علل ریشه‌ای مسئلۀ طراحی اپلیکیشن رزومه ساز نیز وجود دارد که در ادامه به برخی دیگر از آن ها اشاره خواهیم کرد:
	
	نمودار پارتو (\lr{pareto}):
	
	با استفاده از نمودار پارتو می‌توان عواملی که بیشترین تأثیر ممکن را بر روی کیفیت رابط کاربری و تجربه کاربری دارند، مشخص کرد.
	
	به طور مثال ممکن است که نقدها و بازخوردهای کاربران 80 درصد مشکلات اصلی را تشکیل دهد.
	
	نمودار استخوان ماهی (\lr{fish-bone}):
	
	این نمودار به ما کمک می‌کند عوامل مختلفی که ممکن است در نادرست بودن تشخیص و فهم رابط کاربری و تجربه کاربری دخیل باشند را مشخص کنیم.
	
	این عوامل ممکن است مربوط به دستگاه‌ها (مثل موبایل و تبلت)، محیط (شامل نور و رنگ‌ها)، متون و ترتیب اطلاعات (اینکه به طور منظم سامان دهی شدند یا نه) و... باشد.
	
	نمودار پراکندگی (\lr{scatter plot diagram}):
	
	با استفاده از این نمودار می‌توان رابطۀ بین بازخوردهای کاربران با کیفیت رابط کاربری و تجربه کاربری را بررسی کرد. 
	
	در این نمودار به سوالاتی مانند سوال زیر به دقت پاسخ داده می‌شود:
	
	آیا افرادی که نقدها را می‌دهند، مشکلات مشابهی را تجربه می‌کنند؟
	
	تحلیل موضوع عدم موفقیت و تأثیرات آن (\lr{fmea}):
	
	این روش می‌تواند در تعیین اولویت اقداماتی که قرار است برای بهبود رابط کاربری و تجربه کاربری انجام دهیم، کمک کند که بفهمیم کدام یک از اقدامات ضروری‌تر و بهتر است و بیشتر موجب بهبود می‌شود.
	
	به طور مثال تعیین ریسک‌هایی که ممکن است تجربۀ کاربری را تحت تأثیر قرار دهد و انجام اقداماتی برای پیشگیری از آن، از مسائل مربوط به روش تحلیل موضوع عدم موفقیت می‌باشد.
	
	در نهایت، با توجه به این مثال‌ها، تحلیل علت ریشه‌ای در مسائل مرتبط با طراحی اپلیکیشن رزومه ساز و شناسایی عوامل اصلی که تأثیر مستقیم بر کیفیت رابط کاربری و تجربه کاربری دارند، می‌تواند در ارائه راه‌حل‌های دقیق و کارآمد مفید باشند و به بهبود و توسعه اپلیکیشن کمک کنند.
	
\end{document}




